\documentclass[a4paper,11pt,bibliography=totoc]{scrreport}

\usepackage{hyperref}
%\usepackage[nottoc]{tocbibind}
\usepackage[acronyms,toc,sort=standard]{glossaries}

\makeglossaries
\makeindex

\loadglsentries{glossary.tex}

\title{ALMAGESTO - Technical Manual}
\subtitle{A Free Pascal Astronomical Library}
\author{João Marcelo S. Vaz}
\date{\today}


\begin{document}
	
%\frontmatter
\maketitle
\pagenumbering{roman}
\begin{quote}    
Copyright (C) 2020-2021 João Marcelo S. Vaz
\end{quote}
\begin{quote} 
Permission is granted to copy, distribute and/or modify this document under the terms of the GNU Free Documentation License, Version 1.3 or any later version published by the Free Software Foundation; with no Invariant Sections, no Front-Cover Texts, and no Back-Cover Texts. A copy of the license is included in the appendix~\ref{License} entitled ``GNU Free Documentation License''.
\end{quote}
	
\tableofcontents
\printglossary[type=\acronymtype,title=List of Abbreviations]

	
\clearpage
\pagenumbering{arabic}
%\mainmatter
\chapter{Introduction}\label{Introduction}

Almagesto is a free source-code library in pascal that provides common astronomical and astrometric quantities and coordinate systems transformations.

The algorithms used in Almagesto are based on a vector and matrix formulation that is rigorous and consistent with the recommendations by the \gls{IAU} and with the conventions of the \gls{IERS}.

The following sections describe the calculation methods that implement the models used in fundamental astronomy.

\chapter{Coordinate Systems}\label{CoordinateSystems}

\chapter{Date and Time}\label{TimeScales}

The astronomical computations use two main time scales, \gls{UT1} and \gls{TDB}. Both can be derived from \gls{UTC}.

\gls{UTC} is the primary time standard by which the world regulates clocks and time and can be obtained from \gls{StandardTime}, \gls{gls-TZ} and \gls{gls-DST}.

The difference between \gls{UT1} and \gls{UTC} is derived from observation and the uniform time
scale \gls{UTC}. Forecast values are published at \gls{IERSBulletinA} and daily final values at 0:00 UT of UT1-UTC can obtained from \gls{IERSBulletinB} with the following time span: one month with final values and one month with preliminary values. Weekly updated values of DUT1 with 0.1 s precision are announced at \gls{IERSBulletinD}. The latest \gls{IERS} Bulletins can be accessed at \url{https://www.iers.org/IERS/EN/Publications/Bulletins/bulletins.html}.

\gls{UTC} is based on \gls{TAI}, with \glspl{LeapSecond} added to keep it within 0.9 second of \gls{UT1}. \gls{IERS} publishes announcements every six months, whether leap seconds are to occur or not, in its \gls{IERSBulletinC}. All \glspl{LeapSecond} can be obtained from \url{https://hpiers.obspm.fr/eoppc/bul/bulc/UTC-TAI.history}\footnote{As of January 2021, there have been 27 \glspl{LeapSecond} in total, all positive, putting \gls{UTC} 37 seconds behind \gls{TAI}.}.

\gls{UT1R} and \gls{UT2} can be obtained from \gls{UT1}:





\chapter{Coordinate Systems Transformations}\label{CoordinateSystemsTransformations}

\chapter{Ephemerides and Star Catalogs}\label{EphemeridesAndStarCatalogs}

\chapter{Map Projections}\label{MapProjections}

\chapter{Calendars}\label{Calendars}

\appendix
\chapter{Conversion Constants}\label{Conversion Constants}

\chapter{IAU Resolutions}\label{IAU Resolutions}
	
\chapter{GNU Free Documentation License}\label{License}
\input{fdl-1.3}

%\backmatter
\bibliographystyle{naturemag} %apalike
\bibliography{techman}
\printglossary

\end{document}