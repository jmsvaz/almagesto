\documentclass[a4paper,11pt]{scrreport}

\usepackage{hyperref}
\usepackage[nottoc]{tocbibind}
\usepackage[acronyms,toc]{glossaries}

\makeglossaries
\makeindex

\loadglsentries{glossary.tex}

\title{ALMAGESTO - Technical Manual}
\subtitle{A Free Pascal Astronomical Library}
\author{João Marcelo S. Vaz}
\date{\today}


\begin{document}
	
%\frontmatter
\maketitle
\pagenumbering{roman}
\begin{quote}    
Copyright (C) 2020-2021 João Marcelo S. Vaz
\end{quote}
\begin{quote} 
Permission is granted to copy, distribute and/or modify this document under the terms of the GNU Free Documentation License, Version 1.3 or any later version published by the Free Software Foundation; with no Invariant Sections, no Front-Cover Texts, and no Back-Cover Texts. A copy of the license is included in the appendix~\ref{License} entitled ``GNU Free Documentation License''.
\end{quote}
	
\tableofcontents
\printglossary[type=\acronymtype,title=List of Abbreviations]

	
\clearpage
\pagenumbering{arabic}
%\mainmatter
\chapter{Introduction}\label{Introduction}

Almagesto is a free source-code library in pascal that provides common astronomical and astrometric quantities and coordinate systems transformations.

The algorithms used in Almagesto are based on a vector and matrix formulation that is rigorous and consistent with the recommendations by the \gls{IAU} and with the conventions of the \gls{IERS}.

The following sections describe the calculation methods that implement the models used in fundamental astronomy.

\gls{maths}

\chapter{Coordinate Systems}\label{CoordinateSystems}
\chapter{Time Scales}\label{TimeScales}
\chapter{Coordinate Systems Transformations}\label{CoordinateSystemsTransformations}
\chapter{Ephemerides and Star Catalogs}\label{EphemeridesAndStarCatalogs}
\chapter{Map Projections}\label{MapProjections}

\nocite{*}% only for demo to get all entries from the bib data file

\appendix
\chapter{Conversion Constants}\label{Conversion Constants}

\chapter{IAU Resolutions}\label{IAU Resolutions}
	
\chapter{GNU Free Documentation License}\label{License}
\input{fdl-1.3}

%\backmatter
\bibliographystyle{apalike}
\bibliography{techman}
\printglossary

\end{document}