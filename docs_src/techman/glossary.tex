\newacronym{IAU}{IAU}{International Astronomical Union}
\newacronym{IERS}{IERS}{International Earth Rotation and Reference Systems Service}
\newacronym{JD}{JD}{Julian Date}
\newacronym{MJD}{MJD}{Modified Julian Date}

\newacronym{TDB}{TDB}{Barycentric Dynamical Time}
\newacronym{TDT}{TDT}{Terrestrial Dynamical Time}
\newacronym{TT}{TT}{Terrestrial Time}
\newacronym{TCB}{TCB}{Barycentric Coordinate Time}
\newacronym{TCG}{TCG}{Geocentric Coordinate Time}
\newacronym{BCRS}{BCRS}{Barycentric Celestial Reference System}
\newacronym{GCRS}{GCRS}{Geocentric Celestial Reference System}
\newacronym{ICRS}{ICRS}{International Celestial Reference System}
\newacronym{ICRF}{ICRF}{International Celestial Reference Frame}
\newacronym{CIRS}{CIRS}{Celestial Intermediate Reference System}
\newacronym{TIRS}{TIRS}{Terrestrial Intermediate Reference System}
\newacronym{ITRS}{ITRS}{International Terrestrial Reference System}
\newacronym{ITRF}{ITRF}{International Terrestrial Reference Frame}
\newacronym{GTRS}{GTRS}{Geocentric Terrestrial Reference System}
\newacronym{CEO}{CIO}{Celestial Intermediate Origin}
\newacronym{CIP}{CIP}{Celestial Intermediate Pole}
\newacronym{TIO}{TIO}{Terrestrial Intermediate Origin}
\newacronym{ERA}{ERA}{Earth Rotation Angle}
\newacronym{GMST}{GMST}{Greenwich Mean Sidereal Time}
\newacronym{GST}{GST}{Greenwich Sidereal Time}
\newacronym{GAST}{GAST}{Greenwich Apparent Sidereal Time}
\newacronym{EOP}{EOP}{Earth Orientation Parameters}

\newacronym[see={gls-TZ}]{TZ}{TZ}{Time Zone\glsadd{gls-TZ}}
\newacronym[see={gls-DST}]{DST}{DST}{Daylight Saving Time\glsadd{gls-DST}}
\newacronym[see={gls-UT}]{UT}{UT}{Universal Time\glsadd{gls-UT}}
\newacronym[see={gls-UTC}]{UTC}{UTC}{Coordinated Universal Time\glsadd{gls-UTC}}
\newacronym[see={gls-TAI}]{TAI}{TAI}{International Atomic Time\glsadd{gls-TAI}}




\newglossaryentry{StandardTime} {
	name={Standard Time},
	description={The synchronization of clocks within a geographical area or region to a single time standard, rather than using solar time or a locally chosen meridian (longitude) to establish a local mean time standard. Generally, standard time agrees with the local mean time at some meridian that passes through the region, often near the center of the region. Historically, the concept was established during the 19th century to aid weather forecasting and train travel. Applied globally in the 20th century, the geographical areas became extended around evenly spaced meridians into time zones which (usually) centered on them. The standard time set in each time zone has come to be defined in terms of offsets from \gls{UTC}. In regions where \gls{gls-DST} is used, that time is defined by another offset, from the standard time in its applicable \gls{gls-TZ}~\cite{StandardTime2020}},
}

\newglossaryentry{gls-TZ} {
	name={Time Zone},
	description={A designated area of the globe that observes a uniform \gls{StandardTime} for legal, commercial and social purposes. Time zones tend to follow the boundaries of countries and their subdivisions instead of strictly following longitude because it is convenient for areas in close commercial or other communication to keep the same time~\cite{TimeZone2021}},
}

\newglossaryentry{gls-DST} {
	name={Daylight Saving Time},
	description={The practice of advancing clocks during warmer months so that darkness falls later each day according to the clock. The typical implementation of Daylight Saving Time is to set clocks forward by one hour in the spring ("spring forward") and set clocks back by one hour in autumn ("fall back", from the North American English word "fall" for autumn) to return to \gls{StandardTime}. As a result, there is one 23-hour day in late winter or early spring and one 25-hour day in the autumn~\cite{DaylightSavingTime2021}},
}

\newglossaryentry{gls-UT} {
	name={Universal Time},
	description={\acrshort{UT} is a time standard based on Earth's rotation. There are several versions of Universal Time, which differ by up to a few seconds. The most commonly used are \gls{UTC} and \gls{UT1}. All of these versions of Universal Time, except for \gls{UTC}, are based on Earth's rotation relative to distant celestial objects (stars and quasars), but with a scaling factor and other adjustments to make them closer to solar time~\cite{UniversalTime2020}},
}

\newglossaryentry{gls-UTC} {
	name={Coordinated Universal Time},
	description={\acrshort{UTC} is the primary time standard by which the world regulates clocks and time. It is within about 1 second of mean solar time at 0° longitude, and is not adjusted for \gls{gls-DST}. It is effectively a successor to Greenwich Mean Time (GMT). The current version of \gls{UTC} is defined by International Telecommunication Union Recommendation (ITU-R TF.460-6), Standard-frequency and time-signal emissions, and is based on \gls{TAI} with leap seconds added at irregular intervals to compensate for the slowing of the Earth's rotation. Leap seconds are inserted as necessary to keep \gls{UTC} within 0.9 second of the \gls{UT1} variant of universal time~\cite{CoordinatedUniversalTime2020}},
}

\newglossaryentry{UT0} {
	name={UT0},
	description={\gls{gls-UT} determined at an observatory by observing the diurnal motion of stars or extragalactic radio sources, and also from ranging observations of the Moon and artificial Earth satellites. The location of the observatory is considered to have fixed coordinates in a terrestrial reference frame (such as the \gls{ITRF}) but the position of the rotational axis of the Earth wanders over the surface of the Earth; this is known as polar motion. UT0 does not contain any correction for polar motion. The difference between UT0 and \gls{UT1} is on the order of a few tens of milliseconds. The designation UT0 is no longer in common use~\cite{UniversalTime2020}},
}

\newglossaryentry{UT1} {
	name={UT1},
	description={The principal form of \gls{gls-UT}. While conceptually it is mean solar time at 0° longitude, precise measurements of the Sun are difficult. Hence, it is computed from determining the positions of distant quasars using long baseline interferometry, laser ranging of the Moon and artificial satellites, as well as the determination of GPS satellite orbits. UT1 is the same everywhere on Earth, and is proportional to the rotation angle of the Earth with respect to distant quasars, specifically, the \gls{ICRF}, neglecting some small adjustments. The observations allow the determination of a measure of the Earth's angle with respect to the \gls{ICRF}, called the \gls{ERA}~\cite{UniversalTime2020}},
}

\newglossaryentry{UT1R} {
	name={UT1R},
	description={A smoothed version of \gls{UT1}, filtering out periodic variations due to tides. It includes 62 smoothing terms, with periods ranging from 5.6 days to 18.6 years~\cite{UniversalTime2020}},
}

\newglossaryentry{UT2} {
	name={UT2},
	description={A smoothed version of \gls{UT1}, filtering out periodic seasonal variations. It is mostly of historic interest and rarely used anymore~\cite{UniversalTime2020}},
}

\newglossaryentry{gls-TAI} {
	name={International Atomic Time},
	description={\acrshort{TAI} is a high-precision atomic coordinate time standard based on the notional passage of proper time on Earth's geoid. It is the principal realisation of \gls{TT} (with a fixed offset of epoch). It is also the basis for \gls{UTC}~\cite{InternationalAtomicTime2020}},
}

\newglossaryentry{LeapSecond} {
	name={Leap Second},
	description={It's a one-second adjustment that is occasionally applied to \gls{UTC}, to accommodate the difference between precise time (as measured by atomic clocks) and imprecise observed solar time (known as \gls{UT1} and which varies due to irregularities and long-term slowdown in the Earth's rotation)~\cite{LeapSecond2021}},
}

\longnewglossaryentry{IERSBulletinA}
	{name=IERS Bulletin A}
	{\gls{IERS} ''Bulletin A''~\cite{iersBULLETINAProductMetadata} contains Earth orientation parameters x/y pole, UT1-UTC and their errors at daily intervals and predictions for 1 year into the future. Contents are divided into four sections:
	\begin{enumerate}  
		\item General information including key definitions and the most recently adopted values of DUT1 and TAI-UTC.
		\item Quick-look daily estimates of the EOPs determined by smoothing the observed data. This involves the application of systematic corrections and statistical weighting. The results are published with a delay of about one to three days between the date of publication and the last available date with estimated EOP. 
		\item Predictions of x, y, and UT1-UTC, up to 365 days following the last day of data. The predictions use similar algorithms based on seasonal filtering and autoregressive processing for x, y, and UT1.
		\item The combination series for the celestial pole offsets. Bulletin A contains celestial pole offsets with respect to the IAU1980 Nutation theory (dpsi and deps) and the IAU 2000 Resolutions (dX and dY), beginning on 1 January 2003
	\end{enumerate}	}

\longnewglossaryentry{IERSBulletinB}
{name=IERS Bulletin B}
{\gls{IERS} ''Bulletin B''~\cite{iersBULLETINBProductMetadata} provides current information on the Earth's orientation in the IERS Reference System, including Universal Time, coordinates of the terrestrial pole, and celestial pole offsets. Contents are divided into five sections:
	\begin{enumerate}  
		\item Daily final values at 0:00 UT of x, y, UT1-UTC, dX, dY, and their uncertainties. Time span: one month with final values, one month with preliminary values.
		\item Daily final values at 0:00 UT of celestial pole offsets dPsi and dEps in the IAU 1980 system and their uncertainties.
		\item Earth angular velocity (daily estimates of LOD and OMEGA with their uncertainties).
		\item Information on the time scales and announcement of occurring leap seconds.
		\item Average formal precision of the individual and combined series contributing or not to the combination and their agreement with the combination.
	\end{enumerate}	
}

\newglossaryentry{IERSBulletinC} {
	name={IERS Bulletin C},
	description={\gls{IERS} ''Bulletin C'~\cite{iersBULLETINCProductMetadata} provides announcement of leap seconds in UTC and information on UTC-TAI},
}

\newglossaryentry{IERSBulletinD} {
	name={IERS Bulletin D},
	description={\gls{IERS} ''Bulletin D''~\cite{iersBULLETINDProductMetadata} provides announcements of the value of DUT1 to be transmitted with time signals with a precision of +/-0.1s},
}
