\documentclass[a4paper,11pt]{scrreport}

\usepackage{hyperref}

\title{ALMAGESTO - Technical Manual}
\subtitle{A Free Pascal Astronomical Library}
\author{João Marcelo S. Vaz}
\date{\today}


\begin{document}
%\frontmatter
\maketitle
\pagenumbering{roman}
\begin{quote}    
Copyright (C) 2020-2021 João Marcelo S. Vaz
\end{quote}
\begin{quote} 
Permission is granted to copy, distribute and/or modify this document under the terms of the GNU Free Documentation License, Version 1.3 or any later version published by the Free Software Foundation; with no Invariant Sections, no Front-Cover Texts, and no Back-Cover Texts. A copy of the license is included in the appendix~\ref{License} entitled ``GNU Free Documentation License''.
\end{quote}
	
\tableofcontents

	
	
	
\clearpage
\pagenumbering{arabic}
%\mainmatter
\chapter{Introduction}

Almagesto is a free source-code library in pascal that provides common astronomical and astrometric quantities and transformations. The algorithms used in Almagesto are based on a vector and matrix formulation that is rigorous and consistent with the recommendations by the International Astronomical Union (IAU) and IERS conventions.

The following sections describe the calculation methods that implement the models used in fundamental astronomy.

\chapter{Coordinate Systems}
\chapter{Time Scales}
\chapter{Coordinate Systems Transformations}
\chapter{Ephemerides and Star Catalogs}
\chapter{Map Projections}


\appendix
\chapter{Conversion Constants}

\chapter{IAU Resolutions}
	
\chapter{GNU Free Documentation License}\label{License}


%\backmatter
\bibliographystyle{apalike}
\bibliography{techman}

\end{document}